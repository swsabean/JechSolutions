\chapter{Ordinal Numbers}

\Exercise{1} The relation "$(P, <)$ is isomorphic to $(Q, <)$" is an 
equivalence relation (on the class of all partially ordered sets).
\begin{solution}
Let $(P, <)$ be a partially ordered set. Let $\text{id}_P : P \rightarrow P$ 
be the identity function, i.e., $\text{id}_P(p) = p$ for every $p \in P$. 
Clearly, $\text{id}_P$ is a one-to-one function of $P$ onto itself. Also, 
$\text{id}_P$ is obviously order-preserving, and since 
$\text{id}_P = \text{id}^{-1}_P$, it follows that $\text{id}^{-1}_P$ is 
order-preserving. Therefore, $\text{id}_P$ is an automorphism of $(P, >)$. 
Hence, $(P, >)$ is isomorphic to itself.

Let $P(, <)$ be isomorphic to $Q(, <)$ and let $f : P \rightarrow Q$ be an 
isomorphism. Then $f^{-1} : Q \rightarrow P$ is one-to-one, onto, and 
order-preserving. $f^{-1}$ is therefore an isomorphism, and hence, $(Q, <)$ is 
isomorphic to $(P, <)$.

Let $(P, <)$ be isomorphic to $(Q, <)$ and $(Q, <)$ be isomorphic to $(R, <)$. 
Then there exist isomorphisms $f : P \rightarrow Q$ and $g : Q \rightarrow R$. 
Therefore the composition $g \circ f$ is one-to-one and onto since $f$ and $g$ 
are. Furthermore, $g \circ f$ is order-preserving, since $x > y$ implies 
$f(x) > f(y)$ which further implies $g(f(x)) > g(f(y))$. The inverse 
composition, $f^{-1} \circ g^{-1}$ is also order-preserving. Hence $g \circ f$ 
is an isomorphism and thus $(P, <)$ is isomorphic to $(R, <)$.
\end{solution}

\Exercise{2} $\alpha$ is a limit ordinal if and only if $\beta < \alpha$ implies 
$\beta + 1 < \alpha$, for every $\beta$.
\begin{solution}
Let $\alpha$ be a limit ordinal. If $\alpha = 0$, then there is no $\beta$ such
that $\beta < \alpha$. On the other hand, if $\alpha \neq 0$, then let $\beta$ 
be an ordinal such that $\beta < \alpha \leq \beta + 1$. Let $x \in \beta + 1$. 
Then, either $x \in \beta$, whereupon $x \in \alpha$, or $x = \beta$, where 
again we have $x \in \alpha$. Hence, $x \in \beta + 1$ implies $x \in \alpha$, 
or, $\beta + 1 \subset \alpha$. If $\beta + 1 = \alpha$, then $\alpha$ is a
successor ordinal and therefore not a limit ordinal, which is a contradiction. 
If $\beta + 1 \neq \alpha$, then by Lemma 2.11(iii), we have 
$\beta + 1 < \alpha$, which is also a contradiction and we therefore conclude 
that $\beta < \alpha$ implies $\beta + 1 < \alpha$

Conversely, suppose that $\alpha$ is not a limit ordinal, i.e., $\alpha$ is a
successor ordinal. Then $\alpha = \beta + 1$ for some $\beta$. Clearly, 
$\beta < \alpha$. However, $\beta + 1 \nless \alpha$.
\end{solution}

\Exercise{3} If a set $X$ is inductive, then $X \cap \Ord$ is inductive. The set 
$\N = \bigcap \{ X : X \text{ is inductive} \}$ is the least limit ordinal 
$\neq 0$.
\begin{solution}
Let $X$ be an inductive set. Then, by the Separation Schema, we have that 
$X \cap \Ord$ is a set. Since $X$ is inductive, $\emptyset \in X$. Clearly, 
$0 = \emptyset \in \Ord$. Hence $\emptyset \in X \cap \Ord$. Let 
$x \in X \cap \Ord$. Since $X$ is inductive, $x + 1 = x \cap \{ x \} \in X$. 
Using (2.5), we have $x + 1 \in \Ord$. Therefore $x + 1 \in X \cap \Ord$. 
Hence, $X \cap \Ord$ is inductive. 

By Exercise 1.3, $\N$ is transitive. By Exercise 1.7, $(\N, \in)$ is 
well-founded. Since $X \cap \Ord$ is inductive, we have 
$\N \subset X \cap \Ord \subset \Ord$. Therefore, by Lemma 2.11(iv), it follows
that $\N$ is linearly ordered by $\in$. Hence, $\N$ is transitive and 
well-ordered by $\in$, that is, $\N$ is an ordinal. Since $\emptyset \in \N$, 
we have $\N \neq 0$. Let $n \in \N$ be a nonzero ordinal. Then, by Exercise 
1.8, there exists $m \in \N$ such that $n = m + 1$. Thus, $n$ is a successor 
ordinal. Therefore, by Exercise 2.2, $\N$ is a limit ordinal, in fact, it is 
the least nonzero limit ordinal.
\end{solution}

\Exercise{4} (Without the Axiom of Infinity). Let $\omega =$ least limit 
$\alpha \neq 0$ if it exists, $\omega = \Ord$ otherwise. Prove that the 
following statements are equivalent:

\begin{enumerate}
\renewcommand{\labelenumii}{(\roman{enumii})}
\item
There exists an inductive set.
\item
There exists an infinite set.
\item
$\omega$ is a set.

[For (ii) $\rightarrow$ (iii), apply Replacement to the set of all finite 
subsets of $X$.]
\end{enumerate}
\begin{solution}
(i) $\rightarrow$ (ii). Suppose that there exists an inductive set. Let 
$\N = \bigcap \{ X : X$ is inductive$\}$. By Exercise 1.11, $\N$ is T-infinite.
Hence, by Exercises 1.12 and 1.13, $\N$ is infinite.

(ii) $\rightarrow$ (iii). Suppose that there exists an infinite set, $X$, and 
that $\omega = \Ord$. Let $Y = \{ y \in P(Y) : y \text{ is finite} \}$. If 
$y \in Y$, then $y$ is finite; therefore there exists a one-to-one mapping, 
$f$, of $y$ onto some $n \in \omega$. By Replacement, we have $f(Y)$ is a set. 
Clearly $\emptyset \in Y$ and $f(\emptyset) = 0$. Thus, $f(Y)$ is nonempty, 
since $0 \in f(Y)$. Suppose that $n \in f(Y)$. Then there exists $y \in Y$ such
that $f(y) = n$. Since $X$ is infinite, it follows that $X - y$ is nonempty. 
Hence, let $z \in X - y$. Therefore, $y \cup \{ z \} \in Y$ and thus 
$f(y \cup \{ z \}) = n + 1$. It follows that $n + 1 \in f(Y)$. Therefore, by 
induction, $f(Y) = \omega$, and we have $\omega$ is a set

(iii) $\rightarrow$ (i). Let $\omega$ be a set. Then $0 < \omega$, and for 
every $n \in \omega$, $n + 1 \in \omega$ since $\omega$ is a limit ordinal. 
Hence $\omega$ is inductive.
\end{solution}

\Exercise{5} If $W$ is a well-ordered set, then there exists no sequence 
$\langle a_n : n \in \N \rangle$ in $W$ such that $a_0 > a_1 > a_2 > \ldots$.
\begin{solution}
Suppose that there exists such a sequence. Let $X = \{ a_n : n \in \N \}$. Then
$X$ is a nonempty subset of $W$. Therefore, let $x$ be the least element of 
$X$. Let $n$ be the least $n \in \N$ such that $a_n = x$. Hence, 
$a_{n + 1} \geq a_n$. We have therefore reached a contradiction, and conclude 
that no such sequence exists.
\end{solution}

\Exercise{6} There are arbitrarily large limit ordinals; i.e., 
$\forall \alpha \; \exists \beta > \alpha$ ($\beta$ is a limit.)

[Consider $\lim_{n \rightarrow \omega} \alpha_n$, where 
$\alpha_{n+1} = \alpha_n + 1$.]
\begin{solution}
Suppose that there exists a largest limit ordinal, and let $\alpha_0$ be this 
ordinal. For every $n < \omega$, let $\alpha_{n+1} = \alpha_n + 1$. Since 
$\langle \alpha_n : n < \omega \rangle$ is clearly a nondecreasing sequence of 
ordinals, we may define the limit 
$\lim_{n \rightarrow \omega} \alpha_n = \sup \{ \alpha_n : n < \omega \}$. By 
(2.4), $\sup \{ \alpha_n : n < \omega \}$ is an ordinal, hence, let 
$\alpha_{\omega} = \sup \{ \alpha_n : n < \omega \}$. If $\alpha_{\omega}$ is 
a successor, then there exists an ordinal $\xi < \alpha_{\omega}$ such that 
$\xi + 1 = \alpha_{\omega}$.. However, if $\xi < \alpha_{\omega}$, then 
$\xi \in \alpha_{\omega}$, and it follows that $\xi + 1 \in \alpha_{\omega}$. 
We have therefore reached a contradiction and conclude that $\alpha_{\omega}$ 
is a limit ordinal. Since $\alpha_0 < \alpha_{\omega}$, we see that $\alpha_0$ 
is not the largest limit ordinal, which is yet another contradiction. Hence, we
find that there indeed arbitrarily large limit ordinals.
\end{solution}

\Exercise{7} Every normal sequence 
$\langle \gamma_{\alpha} : \alpha \in \Ord \rangle$ has arbitrarily large 
\textit{fixed points}, i.e., $\alpha$ such that $\gamma_{\alpha} = \alpha$.

[Let $\alpha_{n + 1} = \gamma_{\alpha_n}$, and 
$\alpha = \lim_{n \rightarrow \omega} \alpha_n$.]
\begin{solution}
First, since it is not done in the text, we state explicitly that $\Ord$ is 
well-ordered as a class. That is, if $C$ is any nonempty class of ordinals, 
then $C$ has a least element. By virtue of remark (2.3) following Lemma 2.11,
 $\bigcap C$ is the least element of $C$.

Given this declaration, it is necessary to verify the following extension of 
Lemma 2.4. If $\langle \gamma_{\alpha} : \alpha \in \Ord \rangle$ is an 
increasing sequence of ordinals, then $\gamma_{\alpha} \geq \alpha$ for every 
$\alpha \in \Ord$. To see this, suppose that the class 
$X = \left \{ \gamma_{\alpha} : \gamma_{\alpha} < \alpha \right \}$ is 
nonempty, and let $\beta$ be the least element of $X$. Then 
$\gamma_{\beta} < \beta$, and since 
$\langle \gamma_{\alpha} : \alpha \in \Ord \rangle$ is an increasing sequence, 
we have $\gamma_{\gamma_{\beta}} < \gamma_{\beta}$. But this contradicts the 
fact that $\beta$ is the least ordinal such that $\gamma_{\alpha} < \alpha$. 
We therefore conclude that $X$ is empty, and therefore that 
$\gamma_{\alpha} \geq \alpha$ for every $\alpha \in \Ord$.

Next, if $\langle \gamma_{\alpha} : \alpha \in \Ord \rangle$ is a normal 
sequence and $X$ is any nonempty set of ordinals, then 
$\gamma_{\sup X} = \sup \left \{ \gamma_{\alpha} : \alpha \in X \right \}$. For 
this, let $\beta = \sup X$. By remark (2.4) following Lemma 2.11, $\beta$ is an 
ordinal. If $\beta = 0$, then $X = \{ 0 \}$, and clearly 
$\gamma_0 = \sup \{ \gamma_0 \}$. If $\beta$ is a successor ordinal, then 
$\beta = \delta + 1$ for some ordinal $\delta$. Since $\beta = \sup X$, we have 
$\beta \geq \xi$ for every $\xi \in X$. Suppose, in addition, that 
$\delta \geq \xi$ for every $\xi \in X$. Then $\delta$ is an upper bound of 
$X$, and since $\delta < \beta$, we have a contradiction of the fact that 
$\beta = \sup X$. We therefore conclude that there exists $\epsilon \in X$ such 
that $\delta < \epsilon$. Since $\beta = \sup X$, we have 
$\beta \geq \epsilon$. Since $\beta = \inf \{ \zeta : \zeta > \delta \}$, we 
have $\epsilon \geq \beta$. Therefore $\beta = \epsilon$, and it follows that 
$\beta \in X$. Since $\langle \gamma_{\alpha} \rangle$ is increasing, it 
follows that $\gamma_{\beta} \geq \gamma_{\xi}$ for every $\xi \in X$. Since 
$\gamma_{\beta} \in \left \{ \gamma_{\alpha} : \alpha \in X \right \}$, it 
follows that 
$\gamma_{\beta} = \sup \left \{ \gamma_{\alpha} : \alpha \in X \right \}$. If 
$\beta$ is a nonzero limit ordinal, then let $\delta$ be an ordinal such that 
$\delta < \beta$. Since $\beta$ is a limit and since $\beta = \sup X$, there 
exists $\epsilon \in X$ such that $\delta < \epsilon$. Since 
$\langle \gamma_{\alpha} \rangle$ is increasing, we have 
$\gamma_{\delta} < \gamma_{\epsilon}$. Hence, 
$\gamma_{\delta} < \sup \left \{ \gamma_{\alpha} : \alpha \in X \right \}$, 
which yields, since $\langle \gamma_{\alpha} \rangle$ is continuous, 
$\gamma_{\beta} = \sup \left \{ \gamma_{\delta} : \delta < \beta \right \} \leq 
  \sup \left \{ \gamma_{\alpha} : \alpha \in X \right \}$. However, since 
$\langle \gamma_{\alpha} \rangle$ increasing and $\beta = \sup X$, we have 
$\sup \left \{ \gamma_{\alpha} : \alpha \in X \right \} \leq \gamma_{\beta}$. 
Therefore, 
$\gamma_{\beta} = \sup \left \{ \gamma_{\alpha} : \alpha \in X \right \}$, as 
desired.

We are now in a position to prove the main result. We define a sequence, 
$\langle \alpha_n : n < \omega \rangle$ recursively, as follows: Let 
$\alpha \in \Ord$ and let $\alpha_0 = \alpha$, and 
$\alpha_{n + 1} = \gamma_{\alpha_n}$. Let 
$\beta = \sup \left \{ \alpha_n : n < \omega \right \}$. Thus, 
$\beta > \alpha$. Moreover, 
$\gamma_{\beta} = \gamma_{\sup \left \{ \alpha_n : n < \omega \right \}} = \sup 
  \left \{ \gamma_{\alpha_n} : n < \omega \right \} = \sup \left \{ 
  \alpha_{n + 1} : n < \omega \right \} = \beta$.

We may then find an even larger fixed point by taking 
$\alpha = \gamma_{\beta}$ and repeating the process. Hence 
$\langle \gamma_{\alpha} \rangle$ has arbitrarily large fixed points.
\end{solution}

\Exercise{8} For all $\alpha$, $\beta$, and $\gamma$,
\begin{enumerate}[label=(\roman*)]
\item
$\alpha \cdot \left ( \beta + \gamma \right ) = \alpha \cdot \beta + \alpha 
  \cdot \gamma$,
\item 
$\alpha^{\beta + \gamma} = \alpha^{\beta} \cdot \alpha^{\gamma}$,
\item
$\left ( \alpha^{\beta} \right )^{\gamma} = \alpha^{\beta \cdot \gamma}$.
\end{enumerate}
\begin{solution}
In each case, we proceed by transfinite induction on $\gamma$.

(i) If $\gamma = 0$ then
\begin{align*}
\alpha \cdot \left ( \beta + 0 \right ) &= \alpha \cdot \beta \\
&= \alpha \cdot \beta + 0 \\
&= \alpha \cdot \beta + \alpha \cdot 0
\end{align*}
If $\gamma + 1$ is a successor ordinal then
\begin{align*}
\alpha \cdot \left ( \beta + \left ( \gamma + 1 \right )\right ) &= \alpha 
  \cdot \left ( \left ( \beta + \gamma \right ) + 1 \right ) \\
&= \alpha \cdot \left ( \beta + \gamma \right ) + \alpha \\
&= \left ( \alpha \cdot \beta + \alpha \cdot \gamma \right ) + \alpha \\
&= \alpha \cdot \beta + \left ( \alpha \cdot \gamma + \alpha \right ) \\
&= \alpha \cdot \beta + \alpha \cdot \left ( \gamma + 1 \right )
\end{align*}
If $\gamma$ is a nonzero limit ordinal then
\begin{align*}
\alpha \cdot \left ( \beta + \gamma \right ) &= \alpha \cdot \lim_{\xi 
  \rightarrow \gamma} \left ( \beta + \xi \right ) \\
&= \lim_{\xi \rightarrow \gamma} \alpha \cdot \left ( \beta + \xi \right ) \\
&= \lim_{\xi \rightarrow \gamma} \alpha \cdot \beta + \alpha \cdot \xi \\
&= \alpha \cdot \beta + \lim_{\xi \rightarrow \gamma} \alpha \cdot \xi \\
&= \alpha \cdot \beta + \alpha \cdot \gamma
\end{align*}

(ii) If $\gamma = 0$ then
\begin{align*}
\alpha^{\beta + 0} &= \alpha^{\beta} \\
&= \alpha^{\beta} \cdot 1 \\
&= \alpha^{\beta} \cdot \alpha^0
\end{align*}
If $\gamma + 1$ is a successor ordinal then
\begin{align*}
\alpha^{\beta + \left( \gamma + 1 \right)} &= \alpha^{\left( \beta + \gamma 
  \right) + 1} \\
&= \alpha^{\beta + \gamma} \cdot \alpha \\
&= ( \alpha^{\beta} \cdot \alpha^{\gamma} ) \cdot \alpha \\
&= \alpha^{\beta} \cdot \left (\alpha^{\gamma} \cdot \alpha \right) \\
&= \alpha^{\beta} \cdot \alpha^{\gamma + 1}
\end{align*}
If $\gamma$ is a nonzero limit ordinal then
\begin{align*}
\alpha^{\beta + \gamma} &= \alpha^{\beta + \lim_{\xi \rightarrow \gamma} \xi} 
  \\
&= \alpha^{\lim_{\xi \rightarrow \gamma} \left( \beta + \xi \right)} \\
&= \lim_{\xi \rightarrow \gamma} \alpha^{\beta + \xi} \\
&= \lim_{\xi \rightarrow \gamma} \left( \alpha^{\beta} \cdot \alpha^{\xi} 
  \right) \\
&= \alpha^{\beta} \cdot \lim_{\xi \rightarrow \gamma} \alpha^{\xi} \\
&= \alpha^{\beta} \cdot \alpha^{\lim_{\xi \rightarrow \gamma} \xi} \\
&= \alpha^{\beta} \cdot \alpha^{\gamma}
\end{align*}

(iii) If $\gamma = 0$ then
\begin{align*}
\left( \alpha^{\beta} \right)^0 &= 1 \\
&= \alpha^0 \\
&= \alpha^{\beta \cdot 0}
\end{align*}
If $\gamma + 1$ is a successor ordinal then
\begin{align*}
\left( \alpha^{\beta} \right)^{\gamma + 1} &= \left( \alpha^{\beta} 
  \right)^{\gamma} \cdot \alpha^{\beta} \\
&= \alpha^{\beta \cdot \gamma} \cdot \alpha^{\beta} \\
&= \alpha^{\beta \cdot \left( \gamma + 1 \right)}
\end{align*}
If $\gamma$ is a nonzero limit ordinal then
\begin{align*}
\left( \alpha^{\beta} \right)^{\gamma} &= \left( \alpha^{\beta} 
  \right)^{\lim_{\xi \rightarrow \gamma} \xi} \\
&= \lim_{\xi \rightarrow \gamma} \left( \alpha^{\beta} \right)^{\xi} \\
&= \lim_{\xi \rightarrow \gamma} \alpha^{\beta \cdot \xi} \\
&= \alpha^{\lim_{\xi \rightarrow \gamma} \beta \cdot \xi} \\
&= \alpha^{\beta \cdot \lim_{\xi \rightarrow \gamma} \xi} \\
&= \alpha^{\beta \cdot \gamma}
\end{align*}
\end{solution}

\Exercise{9}
\begin{enumerate}[label=(\roman*)]
\item
Show that $\left( \omega + 1 \right) \cdot 2 \neq \omega \cdot 2 + 1 \cdot 2$.
\item
Show that $\left( \omega \cdot 2 \right)^2 \neq \omega^2 \cdot 2^2$.
\end{enumerate}
\begin{solution}
(i)
\begin{align*}
\left( \omega + 1 \right) \cdot 2 &= \left( \omega + 1 \right) + \left( \omega
  + 1 \right) \\
&= \left( \omega + \left( 1 + \omega \right) \right) + 1 \\
&= \left( \omega + \omega \right) + 1 \\
&= \omega \cdot 2 + 1 \\
&< \omega \cdot 2 + 2 \\
&= \omega \cdot 2 + 1 \cdot 2
\end{align*}

(ii)
\begin{align*}
\left( \omega \cdot 2 \right)^2 &= \left( \omega \cdot 2 \right) \cdot \left(
  \omega \cdot 2 \right) \\
&= \left( \omega \cdot \left( 2 \cdot \omega \right) \right) \cdot 2 \\
&= \left( \omega \cdot \omega \right) \cdot 2 \\
&= \omega^2 \cdot 2 \\
&< \omega^2 \cdot 4 \\
&= \omega^2 \cdot 2^2
\end{align*}
\end{solution}

\Exercise{10} If $\alpha < \beta$ then $\alpha + \gamma \leq \beta + \gamma$, 
$\alpha \cdot \gamma \leq \beta \cdot \gamma$, and 
$\alpha^{\gamma} \leq \beta^{\gamma}$.
\begin{solution}
In each case, we proceed by induction on $\gamma$. First we consider sums. If 
$\gamma = 0$, we have $\alpha + 0 = \alpha < \beta = \beta + 0$. If 
$\gamma + 1$ is a successor ordinal, then 
$\alpha + \gamma \leq \beta + \gamma < \left( \beta + \gamma \right) + 1$, 
therefore $\alpha + \left( \gamma + 1 \right) = \left( \alpha + \gamma \right)
+ 1 \leq \left( \beta + \gamma \right) + 1 = \beta + \left( \gamma + 1
\right)$. Finally, if $\gamma$ is a nonzero limit ordinal, then for every 
$\xi < \gamma$, we have $\alpha + \xi \leq \beta + \xi$. From this it follows 
that $\lim_{\xi \rightarrow \gamma} \alpha + \xi \leq \lim_{\xi \rightarrow 
\gamma} \beta + \xi$, or, equivalently, $\alpha + \gamma \leq \beta + \gamma$.

Next we consider products. If $\gamma = 0$, then 
$\alpha \cdot 0 = 0 = \beta \cdot 0$. If $\gamma + 1$ is a successor ordinal, 
then since $\alpha < \beta$, by virtue of Lemma 2.25(i) we have 
$\alpha \cdot \left( \gamma + 1 \right) = \alpha \cdot \gamma + \alpha < \alpha
\cdot \gamma + \beta$ and by the preceding result for sums, 
$\alpha \cdot \gamma + \beta \leq \beta \cdot \gamma + \beta = \beta \cdot 
\left( \gamma + 1 \right)$. Therefore, 
$\alpha \cdot \left( \gamma + 1 \right) \leq \beta \cdot \left( \gamma + 1 
\right)$. Finally, if $\gamma$ is a nonzero limit ordinal, then for every 
$\xi < \gamma$, we have $\alpha \cdot \xi \leq \beta \cdot \xi$. From this it 
follows that 
$\lim_{\xi \rightarrow \gamma} \alpha \cdot \xi \leq \lim_{\xi \rightarrow 
\gamma} \beta \cdot \xi$, or, equivalently, 
$\alpha \cdot \gamma \leq \beta \cdot \gamma$.

Lastly we consider exponentials. If $\gamma = 0$, then 
$\alpha^0 = 1 = \beta^0$. If $\gamma + 1$ is a successor ordinal, then since 
$\alpha < \beta$, by Lemma 2.25(iii), we have 
$\alpha^{\gamma + 1} = \alpha^{\gamma} \cdot \alpha < \alpha^{\gamma} \cdot 
\beta$, and by the preceding result for products, 
$\alpha^{\gamma} \cdot \beta \leq \beta^{\gamma} \cdot \beta = \beta^{\gamma + 
1}$. Therefore, $\alpha^{\gamma + 1} \leq \beta^{\gamma + 1}$. Finally, if 
$\gamma$ is a nonzero limit ordinal, then for every $\xi < \gamma$, we have 
$\alpha^{\xi} \leq \beta^{\xi}$. From this it follows that 
$\lim_{\xi \rightarrow \gamma} \alpha^{\xi} \leq \lim_{\xi \rightarrow \gamma} 
\beta^{\xi}$, or, equivalently, $\alpha^{\gamma} \leq \beta^{\gamma}$.
\end{solution}

\Exercise{11} Find $\alpha$, $\beta$, $\gamma$ such that
\begin{enumerate}[label=(\roman*)]
\item 
$\alpha < \beta$ and $\alpha + \gamma = \beta + \gamma$,
\item 
$\alpha < \beta$ and $\alpha \cdot \gamma = \beta \cdot \gamma$,
\item 
$\alpha < \beta$ and $\alpha^{\gamma} = \beta^{\gamma}$.
\end{enumerate}
\begin{solution}
\begin{enumerate}[label=(\roman*)]
\item
$0 + \omega = 1 + \omega = \omega$.
\item
$1 \cdot \omega = 2 \cdot \omega = \omega$.
\item
$2^\omega = 3^\omega = \omega$.
\end{enumerate}
\end{solution}

\Exercise{12} Let $\varepsilon_0 = \lim_{n \rightarrow \omega} \alpha_n$ where 
$\alpha_0 = \omega$ and $\alpha_{n + 1} = \omega^{\alpha_n}$ for all $n$. Show 
that $\varepsilon_0$ is the least ordinal $\varepsilon$ such that 
$\omega^\varepsilon = \varepsilon$.
\begin{solution}
Let $\beta_n = \omega^{\alpha_n}$ for all $n < \omega$, that is, 
$\beta_n = \alpha_{n + 1}$ for all $n$. Then 
$\lim_{n \rightarrow \omega} \beta_n = \lim_{n \rightarrow \omega} 
\alpha_{n + 1} = \lim_{n \rightarrow \omega} \alpha_n = \varepsilon_0$. 
Additionally, $\lim_{n \rightarrow \omega} \beta_n = \lim_{n \rightarrow 
\omega} \omega^{\alpha_n} = \omega^{\lim_{n \rightarrow \omega} \alpha_n} = 
\omega^{\varepsilon_0}$. Hence, $\varepsilon_0 = \omega^{\varepsilon_0}$.

Suppose that there exists an ordinal $\varepsilon < \varepsilon_0$ such that 
$\varepsilon = \omega^\varepsilon$. Since $n < \omega^n$ for all $n < \omega$, 
it follows that $\omega \leq \varepsilon$. Furthermore, by Lemma~2.25(v), 
$\omega < \omega^\omega$. Therefore $\omega < \varepsilon$. Let $n$ be the 
least natural number such that $\varepsilon < \alpha_n$. Since $n > 0$, there 
exists a natural number $m$ such that $m + 1 = n$. Hence, 
$\alpha_n = \omega^{\alpha_m}$. Since $\alpha_m \leq \varepsilon$, we have 
$\alpha_n = \omega^{\alpha_m} \leq \omega^\varepsilon = \varepsilon$, 
which is a contradiction.
\end{solution}