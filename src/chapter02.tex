\chapter{Ordinal Numbers}

\Exercise{1} The relation "$(P, <)$ is isomorphic to $(Q, <)$" is an 
equivalence relation (on the class of all partially ordered sets).
\begin{solution}
Let $(P, <)$ be a partially ordered set. Let $\text{id}_P : P \rightarrow P$ 
be the identity function, i.e., $\text{id}_P(p) = p$ for every $p \in P$. 
Clearly, $\text{id}_P$ is a one-to-one function of $P$ onto itself. Also, 
$\text{id}_P$ is obviously order-preserving, and since 
$\text{id}_P = \text{id}^{-1}_P$, it follows that $\text{id}^{-1}_P$ is 
order-preserving. Therefore, $\text{id}_P$ is an automorphism of $(P, >)$. 
Hence, $(P, >)$ is isomorphic to itself.

Let $P(, <)$ be isomorphic to $Q(, <)$ and let $f : P \rightarrow Q$ be an 
isomorphism. Then $f^{-1} : Q \rightarrow P$ is one-to-one, onto, and 
order-preserving. $f^{-1}$ is therefore an isomorphism, and hence, $(Q, <)$ is 
isomorphic to $(P, <)$.

Let $(P, <)$ be isomorphic to $(Q, <)$ and $(Q, <)$ be isomorphic to $(R, <)$. 
Then there exist isomorphisms $f : P \rightarrow Q$ and $g : Q \rightarrow R$. 
Therefore the composition $g \circ f$ is one-to-one and onto since $f$ and $g$ 
are. Furthermore, $g \circ f$ is order-preserving, since $x > y$ implies 
$f(x) > f(y)$ which further implies $g(f(x)) > g(f(y))$. The inverse 
composition, $f^{-1} \circ g^{-1}$ is also order-preserving. Hence $g \circ f$ 
is an isomorphism and thus $(P, <)$ is isomorphic to $(R, <)$.
\end{solution}

\Exercise{2} $\alpha$ is a limit ordinal if and only if $\beta < \alpha$ implies 
$\beta + 1 < \alpha$, for every $\beta$.
\begin{solution}
Let $\alpha$ be a limit ordinal. If $\alpha = 0$, then there is no $\beta$ such
that $\beta < \alpha$. On the other hand, if $\alpha \neq 0$, then let $\beta$ 
be an ordinal such that $\beta < \alpha \leq \beta + 1$. Let $x \in \beta + 1$. 
Then, either $x \in \beta$, whereupon $x \in \alpha$, or $x = \beta$, where 
again we have $x \in \alpha$. Hence, $x \in \beta + 1$ implies $x \in \alpha$, 
or, $\beta + 1 \subset \alpha$. If $\beta + 1 = \alpha$, then $\alpha$ is a
successor ordinal and therefore not a limit ordinal, which is a contradiction. 
If $\beta + 1 \neq \alpha$, then by Lemma 2.11(iii), we have 
$\beta + 1 < \alpha$, which is also a contradiction and we therefore conclude 
that $\beta < \alpha$ implies $\beta + 1 < \alpha$

Conversely, suppose that $\alpha$ is not a limit ordinal, i.e., $\alpha$ is a
successor ordinal. Then $\alpha = \beta + 1$ for some $\beta$. Clearly, 
$\beta < \alpha$. However, $\beta + 1 \nless \alpha$.
\end{solution}

\Exercise{3} If a set $X$ is inductive, then $X \cap \Ord$ is inductive. The set 
$\N = \bigcap \{ X : X \text{ is inductive} \}$ is the least limit ordinal 
$\neq 0$.
\begin{solution}
Let $X$ be an inductive set. Then, by the Separation Schema, we have that 
$X \cap \Ord$ is a set. Since $X$ is inductive, $\emptyset \in X$. Clearly, 
$0 = \emptyset \in \Ord$. Hence $\emptyset \in X \cap \Ord$. Let 
$x \in X \cap \Ord$. Since $X$ is inductive, $x + 1 = x \cap \{ x \} \in X$. 
Using (2.5), we have $x + 1 \in \Ord$. Therefore $x + 1 \in X \cap \Ord$. 
Hence, $X \cap \Ord$ is inductive. 

By Exercise 1.3, $\N$ is transitive. By Exercise 1.7, $(\N, \in)$ is 
well-founded. Since $X \cap \Ord$ is inductive, we have 
$\N \subset X \cap \Ord \subset \Ord$. Therefore, by Lemma 2.11(iv), it follows
that $\N$ is linearly ordered by $\in$. Hence, $\N$ is transitive and 
well-ordered by $\in$, that is, $\N$ is an ordinal. Since $\emptyset \in \N$, 
we have $\N \neq 0$. Let $n \in \N$ be a nonzero ordinal. Then, by Exercise 
1.8, there exists $m \in \N$ such that $n = m + 1$. Thus, $n$ is a successor 
ordinal. Therefore, by Exercise 2.2, $\N$ is a limit ordinal, in fact, it is 
the least nonzero limit ordinal.
\end{solution}

\Exercise{4} (Without the Axiom of Infinity). Let $\omega =$ least limit 
$\alpha \neq 0$ if it exists, $\omega = \Ord$ otherwise. Prove that the 
following statements are equivalent:

\begin{enumerate}
\renewcommand{\labelenumii}{(\roman{enumii})}
\item
There exists an inductive set.
\item
There exists an infinite set.
\item
$\omega$ is a set.

[For (ii) $\rightarrow$ (iii), apply Replacement to the set of all finite 
subsets of $X$.]
\end{enumerate}
\begin{solution}
(i) $\rightarrow$ (ii). Suppose that there exists an inductive set. Let 
$\N = \bigcap \{ X : X$ is inductive$\}$. By Exercise 1.11, $\N$ is T-infinite.
Hence, by Exercises 1.12 and 1.13, $\N$ is infinite.

(ii) $\rightarrow$ (iii). Suppose that there exists an infinite set, $X$, and 
that $\omega = \Ord$. Let $Y = \{ y \in P(Y) : y \text{ is finite} \}$. If 
$y \in Y$, then $y$ is finite; therefore there exists a one-to-one mapping, 
$f$, of $y$ onto some $n \in \omega$. By Replacement, we have $f(Y)$ is a set. 
Clearly $\emptyset \in Y$ and $f(\emptyset) = 0$. Thus, $f(Y)$ is nonempty, 
since $0 \in f(Y)$. Suppose that $n \in f(Y)$. Then there exists $y \in Y$ such
that $f(y) = n$. Since $X$ is infinite, it follows that $X - y$ is nonempty. 
Hence, let $z \in X - y$. Therefore, $y \cup \{ z \} \in Y$ and thus 
$f(y \cup \{ z \}) = n + 1$. It follows that $n + 1 \in f(Y)$. Therefore, by 
induction, $f(Y) = \omega$, and we have $\omega$ is a set

(iii) $\rightarrow$ (i). Let $\omega$ be a set. Then $0 < \omega$, and for 
every $n \in \omega$, $n + 1 \in \omega$ since $\omega$ is a limit ordinal. 
Hence $\omega$ is inductive.
\end{solution}

\Exercise{5} If $W$ is a well-ordered set, then there exists no sequence 
$\langle a_n : n \in \N \rangle$ in $W$ such that $a_0 > a_1 > a_2 > \ldots$.
\begin{solution}
Suppose that there exists such a sequence. Let $X = \{ a_n : n \in \N \}$. Then
$X$ is a nonempty subset of $W$. Therefore, let $x$ be the least element of 
$X$. Let $n$ be the least $n \in \N$ such that $a_n = x$. Hence, 
$a_{n + 1} \geq a_n$. We have therefore reached a contradiction, and conclude 
that no such sequence exists.
\end{solution}

\Exercise{6} There are arbitrarily large limit ordinals; i.e., 
$\forall \alpha \; \exists \beta > \alpha$ ($\beta$ is a limit.)

[Consider $\lim_{n \rightarrow \omega} \alpha_n$, where 
$\alpha_{n+1} = \alpha_n + 1$.]
\begin{solution}
Suppose that there exists a largest limit ordinal, and let $\alpha_0$ be this 
ordinal. For every $n < \omega$, let $\alpha_{n+1} = \alpha_n + 1$. Since 
$\langle \alpha_n : n < \omega \rangle$ is clearly a nondecreasing sequence of 
ordinals, we may define the limit 
$\lim_{n \rightarrow \omega} \alpha_n = \sup \{ \alpha_n : n < \omega \}$. By 
(2.4), $\sup \{ \alpha_n : n < \omega \}$ is an ordinal, hence, let 
$\alpha_{\omega} = \sup \{ \alpha_n : n < \omega \}$. If $\alpha_{\omega}$ is 
a successor, then there exists an ordinal $\xi < \alpha_{\omega}$ such that 
$\xi + 1 = \alpha_{\omega}$.. However, if $\xi < \alpha_{\omega}$, then 
$\xi \in \alpha_{\omega}$, and it follows that $\xi + 1 \in \alpha_{\omega}$. 
We have therefore reached a contradiction and conclude that $\alpha_{\omega}$ 
is a limit ordinal. Since $\alpha_0 < \alpha_{\omega}$, we see that $\alpha_0$ 
is not the largest limit ordinal, which is yet another contradiction. Hence, we
find that there indeed arbitrarily large limit ordinals.
\end{solution}