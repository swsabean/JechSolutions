\chapter{Ordinal Numbers}

\Exercise{1} The relation "$(P, <)$ is isomorphic to $(Q, <)$" is an 
equivalence relation (on the class of all partially ordered sets).
\begin{solution}
Let $(P, <)$ be a partially ordered set. Let $\text{id}_P : P \rightarrow P$ 
be the identity function, i.e., $\text{id}_P(p) = p$ for every $p \in P$. 
Clearly, $\text{id}_P$ is a one-to-one function of $P$ onto itself. Also, 
$\text{id}_P$ is obviously order-preserving, and since 
$\text{id}_P = \text{id}^{-1}_P$, it follows that $\text{id}^{-1}_P$ is 
order-preserving. Therefore, $\text{id}_P$ is an automorphism of $(P, >)$. 
Hence, $(P, >)$ is isomorphic to itself.

Let $P(, <)$ be isomorphic to $Q(, <)$ and let $f : P \rightarrow Q$ be an 
isomorphism. Then $f^{-1} : Q \rightarrow P$ is one-to-one, onto, and 
order-preserving. $f^{-1}$ is therefore an isomorphism, and hence, $(Q, <)$ is 
isomorphic to $(P, <)$.

Let $(P, <)$ be isomorphic to $(Q, <)$ and $(Q, <)$ be isomorphic to $(R, <)$. 
Then there exist isomorphisms $f : P \rightarrow Q$ and $g : Q \rightarrow R$. 
Therefore the composition $g \circ f$ is one-to-one and onto since $f$ and $g$ 
are. Furthermore, $g \circ f$ is order-preserving, since $x > y$ implies 
$f(x) > f(y)$ which further implies $g(f(x)) > g(f(y))$. The inverse 
composition, $f^{-1} \circ g^{-1}$ is also order-preserving. Hence $g \circ f$ 
is an isomorphism and thus $(P, <)$ is isomorphic to $(R, <)$.
\end{solution}