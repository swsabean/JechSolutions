\part{Basic Set Theory}

\chapter{Axioms of Set Theory}

\Exercise{1} Verify (1.1).
\begin{solution}
  If $a = c$ and $b = d$, then $(a, b) = (c, d)$ as a consequence of an axiom of
  first-order logic with equality, namely, that equals may be substituted for
  equals in a formula.

  Conversely, suppose that $(a, b) = (c, d)$. Then
  \[
    \{ \{ a \}, \{ a, b \} \}  = \{ \{ c \}, \{ c, d \} \}.
  \]
  If $a = b$, then
  \[
    \{ \{ a \}, \{ a, b \} \} = \{ \{ a \}, \{ a, a \} \} = \{ \{ a \} \}.
  \]
  Therefore $\{ a \} = \{ c \}$ and  $\{ a \} = \{ c, d \}$. Hence, $a = c = d$,
  from which it follows that $a = c$ and $b = d$. If $a \neq b$, then
  $\{ a \} = \{ c \}$ and $\{ a, b \} = \{ c, d \}$. Therefore $a = c$, and from
  this it follows that $\{ a, b \} = \{ a, d \}$. Hence $b = d$.
\end{solution}

\Exercise{2} There is no set $X$ such that $P(X) \subset X$.
\begin{solution}
  Suppose there exists a set $X$ such that $P(X) \subset X$. 
  Let $Y = \{ x : x \in X \text{ and } x \notin x \}$. Clearly, $Y \subset X$,
  hence $Y \in P(X)$ and therefore $Y \in X$. However $Y \in Y$ if and only if
  $Y \notin Y$. We have therefore reached a contradiction and conclude that no
  such set $X$ exists.
\end{solution}

Let
\[
  \N = \bigcap \{ X \colon X \text{ is inductive} \}.
\]
$\N$ is the smallest inductive set. Let us use the following notation:
\[
  0 = \emptyset, \quad
  1 = \{ 0 \}, \quad
  2 = \{ 0, 1 \}, \quad
  3 = \{ 0, 1, 2 \}, \quad
  \ldots.
\]
If $n \in \N$, let $ n + 1 = n \cup \{ n \}$. Let us define $<$ (on $\N$) by
$n < m$ if and only if $n \in m$.

A set $T$ is \emph{transitive} if $x \in T$ implies $x \subset T$.

\Exercise{3} If $X$ is inductive, then the set
$\{ x \in X : x \subset X \}$ is inductive. Hence $\N$ is
transitive, and for each $n$, $n = \{ m \in \N : m < n \}$.
\begin{solution}
  Let $X$ be inductive. Let $Y = \{ x \in X : x \subset X \}$.
  Since $X$ is inductive, $\emptyset \in X$. Since 
  $\emptyset \subset X$, $\emptyset \in Y$. Let $x \in Y$, then
  $x \in X$ and $x \subset X$. Since $X$ is inductive, 
  $x \cup \{ x \} \in X$. If $y \in x \cup \{ x \}$, then 
  $y \in x$ or $y = x$. If $y \in x$, then since $x \subset X$,
  we have $y \in X$. If $y = x$, then clearly $y \in X$. Hence 
  $x \cup \{ x \} \subset X$. Thus, it follows that 
  $x \cup \{ x \} \in Y$ and therefore $Y$ is inductive.
  
  Since $\N$ is inductive, the set 
  $M = \{ n \in \N : n \subset \N \}$ is inductive. Clearly, 
  $M \subset \N$, and since $M$ is inductive, $\N \subset M$, 
  and therefore $M = \N$. From this it follows that for every 
  $n \in \N$, $n \subset \N$. Hence, $\N$ is transitive. If 
  $n \in \N$, then $n \subset \N$. Hence if $m \in n$, then 
  $m \in \N$, and by definition, $m < n$. Therefore 
  $n \subset \{ m \in \N : m < n \}$. Conversely, if 
  $k \in \{ m \in \N : m < n \}$, then $k < n$ and consequently
  $k \in n$. It follows that $\{ m \in \N : m < n \} \subset n$
  and therefore that $n = \{ m \in \N : m < n \}$.
\end{solution}

\Exercise{4} If $X$ is inductive, then the set 
$\{ x \in X : x \text{ is transitive} \}$ is inductive. Hence
every $n \in \N$ is transitive.
\begin{solution}
  Let $X$ be inductive. Let $Y = \{ x \in X : x \text{ is
  transitive} \}$. Since $X$ is inductive, $\emptyset \in X$.
  Since $\emptyset$ is transitive (vacuously), 
  $\emptyset \in Y$. Let $x \in Y$, then $x \in X$ and $x$ is
  transitive. Since $X$ is inductive, $x \cup \{ x \} \in X$.
  If $y \in x \cup \{ x \}$, then $y \in x$ or $y = x$. If
  $y \in x$, then since $x$ is transitive, $y \subset x$ and
  therefore $y \subset x \cup \{ x \}$. If $y = x$, then
  clearly $ y \subset x \cup \{ x \}$. Hence
  $x \cup \{ x \}$ is transitive, from which it follows that
  $x \cup \{ x \} \in Y$. Therefore $Y$ is inductive.

  Since $\N$ is inductive, the set 
  $M = \{ n \in \N : n \text{ is transitive} \}$ is inductive.
  Clearly, $M \subset \N$, and since $M$ is inductive,
  $\N \subset M$, and therefore $M = \N$. From this it follows
  that for every $n \in \N$, $n$ is transitive.
\end{solution}

\Exercise{5} If $X$ is inductive, then the set
$\{ x \in X : x \text{ is transitive and } x \notin x \}$ is
inductive. Hence $n \notin n$ and $n \neq n + 1$ for each
$n \in \N$.
\begin{solution}
  Let $X$ be inductive. Let
  $Y = \{ x \in X : x \text{ is transitive and } x \notin x \}$.
  Since $X$ is inductive, $\emptyset \in X$. Since $\emptyset$ is
  transitive and $\emptyset \notin \emptyset$, $\emptyset \in Y$.
  Let $x \in Y$, then $x \in X$, $x$ is transitive, and 
  $x \notin x$. Since $X$ is inductive, $x \cup \{ x \} \in X$.
  If $y \in x \cup \{ x \}$, then $y \in x$ or $y = x$. If
  $y \in x$, then since $x$ is transitive, $y \subset x$ and
  therefore $y \subset x \cup \{ x \}$. If $y = x$, then clearly
  $ y \subset x \cup \{ x \}$. Hence $x \cup \{ x \}$ is
  transitive. Suppose that $x \cup \{ x \} \in x \cup \{ x \}$.
  Then $x \cup \{ x \} \in x$ or $x \cup \{ x \} = x$. If
  $x \cup \{ x \} \in x$, then since $x$ is transitive, we have
  $x \cup \{ x \} \subset x$. Therefore, since 
  $x \in x \cup \{ x \}$, we have $x \in x$, which is a
  contradiction. If $x \cup \{ x \} = x$, then since
  $x \in x \cup \{ x \}$, we again have $x \in x$. Thus in either
  case we have reached a contradiction, and conclude that
  $x \cup \{ x \} \notin x \cup \{ x \}$. Therefore
  $x \cup \{ x \} \in Y$. Hence $Y$ is inductive.

  Since $\N$ is inductive, the set
  $M = \{ n \in \N : n \text{ is transitive and } n \notin n \}$
  is inductive. Clearly, $M \subset \N$, and since $M$ is
  inductive, $\N \subset M$, and therefore $M = \N$. Therefore
  $n \notin n$ for every $n \in \N$. Since
  $n \in n \cup \{ n \}$, but $n \notin n$, it follows that
  $n \neq n + 1$.
\end{solution}

\Exercise{6} If $X$ is inductive, then the set $\{ x \in X : x$
is transitive and every nonempty $z \subset x$ has an
$\in$-minimal element$\}$ is inductive ($t$ is
$\in$-\textit{minimal} in $z$ if there is no $s \in z$ such
that $s \in t$).
\begin{solution}
  Let $X$ be inductive. Let $Y = \{ x \in X : x$ is transitive and
  every nonempty $z \subset x$ has an $\in$-minimal element$\}$.
  Since $X$ is inductive, $\emptyset \in X$. Since $\emptyset$ is
  transitive and since $\emptyset$ has no nonempty subsets,
  $\emptyset \in Y$. Let $x \in Y$, then $x \in X$, $x$ is transitive,
  and every nonempty subset of $x$ has an $\in$-minimal element. Since
  $X$ is inductive, $x \cup \{ x \} \in X$. If $y \in x \cup \{ x \}$,
  then $y \in x$ or $y = x$. If $y \in x$, then since $x$ is transitive,
  $y \subset x$ and therefore $y \subset x \cup \{ x \}$. If $y = x$,
  then clearly $ y \subset x \cup \{ x \}$. Hence $x \cup \{ x \}$ is
  transitive. Let $z$ be a nonempty subset of $x \cup \{ x \}$. If 
  $z - \{ x \}$ is empty, then clearly $z = \{ x \}$, hence $z$ has an
  $\in$-minimal element, namely, $x$. If $z - \{ x \}$ is nonempty, then
  since $z - \{ x \} \subset x$, it follows that $z - \{ x \}$ has an
  $\in$-minimal element. Let $t$ be an $\in$-minimal element of
  $z - \{ x \}$. Then $t$ is also an $\in$-minimal element of $z$. To
  see this, suppose that $x \in t$. Since $x$ is transitive, $t \in x$
  implies $t \subset x$. Hence, $x \in t$ implies $x \in x$. This means
  that $x$ has no $\in$-minimal element, which is a contradiction.
  Therefore, $x \cup \{ x \} \in Y$. Hence $Y$ is inductive.
\end{solution}

Exercise{7} Every nonempty $X \subset \N$ has an $\in$-minimal element.
[Pick $n \in X$ and look at $X \cap n$.]
\begin{solution}
  Let $X \subset \N$ be nonempty. Since $X$ is nonempty, let $n \in X$.
  If $n \cap X = \emptyset$, then $m \in n$ implies $m \notin X$. Hence
  $n$ is an $\in$-minimal element of $X$. If $n \cap X$ is nonempty,
  then since $n \cap X$ is a nonempty subset of $n$, by Exercise 1.6,
  $n \cap X$ has an $\in$-minimal element. Let $t$ be an $\in$-minimal
  element of $n \cap X$. Suppose that $s \in X$ such that $s \in t$.
  Since, $n$ is transitive, $t \in n$ implies $t \subset n$, and
  therefore $s \in n$. This contradicts the minimality of $t$.
  Therefore, no such $s$ exists and we conclude that $t$ is an
  $\in$-minimal element of $X$.
  
\end{solution}