\part{Basic Set Theory}

\chapter{Axioms of Set Theory}

\Exercise{1} Verify (1.1).
\begin{solution}
  If $a = c$ and $b = d$, then $(a, b) = (c, d)$ as a consequence of an axiom of
  first-order logic with equality, namely, that equals may be substituted for
  equals in a formula.

  Conversely, suppose that $(a, b) = (c, d)$. Then
  \[
    \{ \{ a \}, \{ a, b \} \}  = \{ \{ c \}, \{ c, d \} \}.
  \]
  If $a = b$, then
  \[
    \{ \{ a \}, \{ a, b \} \} = \{ \{ a \}, \{ a, a \} \} = \{ \{ a \} \}.
  \]
  Therefore $\{ a \} = \{ c \}$ and  $\{ a \} = \{ c, d \}$. Hence, $a = c = d$,
  from which it follows that $a = c$ and $b = d$. If $a \neq b$, then
  $\{ a \} = \{ c \}$ and $\{ a, b \} = \{ c, d \}$. Therefore $a = c$, and from
  this it follows that $\{ a, b \} = \{ a, d \}$. Hence $b = d$.
\end{solution}

\Exercise{2} There is no set $X$ such that $P(X) \subset X$.
\begin{solution}
  Suppose there exists a set $X$ such that $P(X) \subset X$. 
  Let $Y = \{ x : x \in X \text{ and } x \notin x \}$. Clearly, $Y \subset X$,
  hence $Y \in P(X)$ and therefore $Y \in X$. However $Y \in Y$ if and only if
  $Y \notin Y$. We have therefore reached a contradiction and conclude that no
  such set $X$ exists.
\end{solution}

Let
\[
  \N = \bigcap \{ X \colon X \text{ is inductive} \}.
\]
$\N$ is the smallest inductive set. Let us use the following notation:
\[
  0 = \emptyset, \quad
  1 = \{ 0 \}, \quad
  2 = \{ 0, 1 \}, \quad
  3 = \{ 0, 1, 2 \}, \quad
  \ldots.
\]
If $n \in \N$, let $ n + 1 = n \cup \{ n \}$. Let us define $<$ (on $\N$) by
$n < m$ if and only if $n \in m$.

A set $T$ is \emph{transitive} if $x \in T$ implies $x \subset T$.